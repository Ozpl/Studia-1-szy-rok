\documentclass[11pt,a4paper]{article}

\usepackage{polski}
\usepackage[utf8]{inputenc}
\usepackage{amsmath}
\usepackage{mathrsfs}
\usepackage{graphicx}
\graphicspath{ {Obrazy/MMAsprawko2/} }
\usepackage{wrapfig}

\title{Metody matematyczne w automatyce - sprawozdanie nr 2}
\author{Piotr Zakrzewski}
\date{27.03.2018r.}

\begin{document}
\maketitle
Funkcja pierwszego sygnału zapisywana jest w postaci:

\[
    f(t) =\left\{
                \begin{array}{ll}
                  1, dla\ t <0,\pi> \\
                  0, dla\ t \in reszty
                \end{array}
              \right.
\]
\begin{figure}[h]
\centering
\includegraphics[width=13cm]{sygnal1}
\end{figure}

Należy teraz sprawdzić czy sygnał spełnia warunki Dirichleta. Pierwszym warunkiem jest bezwględna całkowalność, którą możemy zapisać jako następujący wzór \ref{warunek1}: \\
\begin{equation}
\int_{-\infty}^{\infty} f(t) dt < \infty
\label{warunek1}
\tag{1.1}
\end{equation}

Należy teraz podstawić nasz sygnał (pomijamy $ \int_{-\pi}^{0} 0 dt $ i $ \int_{\pi}^{2\pi} 0 dt $, bo są równe 0): \\
\begin{equation}
\int_{0}^{\pi} 1 dt = 1 \cdot t \Big|^\pi_0 = \pi - 0 = \pi
\end{equation}

$ \pi $ jest mniejsze od $ \infty $, więc warunek pierwszy jest spełniony. \\

Transformata Fouriera dla sygnału nieokresowego wygląda tak: \\
\begin{equation}
\hat{f} (\omega) = \mathcal{F}[f(t)] = \int_{-\infty}^{\infty} f(t) e^{-j \omega t} dt, \ \ \ \omega \in (-\infty,\infty)
\end{equation}

\begin{align}
\begin{split}
\int_{0}^{\pi} 1 \cdot e^{-j \omega t} dt = &{} 1 \cdot \frac{1}{-j \omega } \cdot e^{-j \omega t} \Big|^\pi_0 = \\
&{} = \frac{-1}{j \omega} \cdot (e^{-j \omega \pi} - e^{-j \omega 0}) = \frac{-1}{j \omega} \cdot (e^{-j \omega \pi} - 1) = \\
&{} = j \cdot \frac{1}{\omega}(e^{-j \omega \pi} - 1) = j \cdot \frac{1}{\omega} \cdot [cos(-\omega t) + j \cdot sin(-\omega t) - 1] = \\
&{} = j \cdot \frac{1}{\omega} \cdot (cos(\omega t) + j \cdot -sin(\omega t) - 1)\\
&{} = -j \cdot \frac{1}{\omega}(1 - cos(\omega \pi) + j \cdot sin(\omega \pi)) = \\
&{} = -j \cdot \frac{1}{\omega} \cdot (2sin^2(\frac{\omega \pi}{2}) + j \cdot sin(\omega \pi)) = \frac{sin(\omega \pi)}{\omega} - j \cdot \frac{2sin^2(\frac{\omega \pi}{2})}{\omega}
\end{split}
\end{align}

\begin{equation}
|\hat{f} (\omega)| = \sqrt{ (\frac{sin(\omega \pi)}{\omega})^2 + (\frac{2sin^2(\frac{\omega \pi}{2})}{\omega})^2 }
\end{equation}

\begin{equation}
arg \hat{f} (\omega) = arctg (\frac{\frac{-2sin^2(\frac{\omega \pi}{2})}{\omega}}{\frac{sin(\omega \pi)}{\omega}}) = arctg (\frac{-2sin^2(\frac{\omega \pi}{2})}{sin(\omega \pi)})
\end{equation}

\begin{equation}
E = \int_{-\infty}^{\infty} f(t)^2 dt = \int_{0}^{\pi} 1^2 dt = t \Big|^\pi_0 = \pi
\end{equation}

\end{document}