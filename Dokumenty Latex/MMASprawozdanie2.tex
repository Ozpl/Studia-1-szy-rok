\documentclass[11pt,a4paper]{article}

\usepackage{polski}
\usepackage[utf8]{inputenc}
\usepackage{amsmath}
\usepackage{mathrsfs}
\usepackage{graphicx}
\usepackage{wrapfig}
\usepackage{relsize}

\newcounter{liczba1}
\newcounter{liczba2}
\setcounter{liczba1}{1}
\setcounter{liczba2}{1}

\numberwithin{liczba1}{liczba2}
\renewcommand{\theequation}{\arabic{liczba1}.\arabic{equation}}
\setcounter{equation}{1}

\graphicspath{ {Obrazy/MMAsprawko2/} }
\newcommand\tab[1][1cm]{\hspace*{#1}}


\title{Metody matematyczne w automatyce - sprawozdanie nr 2}
\author{Analiza sygnałów z czasem ciągłym - sygnały nieokresowe}
\date{13.03.2018r. Piotr Zakrzewski nr. 42033, AiR L2 semestr 2}

\begin{document}
\maketitle
\section{Sygnał nr 1}

\tab Pierwszy sygnał, dla której wyznaczymy transformatę Fouriera wygląda następująco:
\begin{figure}[h]
\centering
\includegraphics[width=13cm]{sygnal1}
\end{figure}

Możemy go zapisać w poniższej postaci:
\[
    f(t) =\left\{
                \begin{array}{ll}
                  1, dla\ t <0,\pi> \\
                  0, dla\ t \in \text{reszty}
                \end{array}
\label{sygnal1klamra}
\tag{1.1}
              \right.
\]

W pierwszej kolejności należy sprawdzić czy sygnał spełnia warunki Dirichleta. Pierwszym warunkiem jest bezwględna całkowalność, którą możemy zapisać jako następujący wzór \ref{warunek1}: \\
\begin{equation}
\int_{-\infty}^{\infty} f(t) dt < \infty
\label{warunek1}
\end{equation}
\pagebreak

Aby to zrobić, musimy podstawić nasz sygnał pod wzór \ref{warunek1} (pomijamy $ \mathlarger{\int_{-\pi}^{0}} 0 \ dt $ i $ \mathlarger{\int_{\pi}^{2\pi}} 0 \ dt $, gdyż obie całki są równe 0): \\
\begin{equation}
\int_{0}^{\pi} 1 dt = 1 \cdot t \ \Bigg|^\pi_0 = \pi - 0 = \pi
\end{equation} \\

Wynika z tego, że $ \pi $ jest mniejsze od $ \infty $, więc warunek pierwszy jest spełniony. Następne dwa warunki: \textit{"funkcja $ f $ w przedziale jednego okresu ma skończoną liczbę maksimów lokalnych i minimów lokalnych"} oraz \textit{"funkcja $ f $ w przedziale jednego okresu posiada skończoną liczbę punktów nieciągłości pierwszego rodzaju"}. Oba warunki są spełnione, a możemy to wywnioskować z rysunku sygnału. \\

Transformata Fouriera: \\
\begin{equation}
\hat{f} (\omega) = \mathcal{F}[f(t)] = \int_{-\infty}^{\infty} f(t) e^{-j \omega t} dt, \ \ \ \omega \in (-\infty,\infty)
\label{transformata}
\end{equation}

W rozwiązywaniu zadań przydadzą nam się następujące wzory:
\begin{gather*}
cos(-x) = cos(x), \ \ \ \ sin(-x) = -sin(x) \\
1 - cos(x) = 2 \cdot sin^2{\frac{x}{2}} \\
e^{jx} = cos(x) + j \cdot sin(x), \ \ \ \ e^{-jx} = cos(-x) + j \cdot sin(-x) \\
\int x e^{ax} dx = e^{ax}\Big[\frac{x}{a} - \frac{1}{a^2} \Big] \text{\ \ \ \ (w sygnale nr 2)}
\end{gather*}

Postawmy naszą funkcję do wzoru na transformatę (\ref{transformata}): \\
\begin{equation}
\begin{split}
\int_{0}^{\pi} 1 \cdot e^{-j \omega t} dt
&{} = 1 \cdot \frac{1}{-j \omega } \cdot e^{-j \omega t} \ \Bigg|^\pi_0 = \frac{-1}{j \omega} \cdot (e^{-j \omega \pi} - e^{-j \omega 0}) \\
&{} = \frac{-1}{j \omega} \cdot (e^{-j \omega \pi} - 1) = j \cdot \frac{1}{\omega}(e^{-j \omega \pi} - 1) \\
&{} = j \cdot \frac{1}{\omega} \cdot [cos(-\omega t) + j \cdot sin(-\omega t) - 1]\\
&{} = j \cdot \frac{1}{\omega} \cdot (cos(\omega t) + j \cdot -sin(\omega t) - 1)\\
&{} = -j \cdot \frac{1}{\omega}(1 - cos(\omega \pi) + j \cdot sin(\omega \pi)) \\
&{} = -j \cdot \frac{1}{\omega} \cdot (2sin^2(\frac{\omega \pi}{2}) + j \cdot sin(\omega \pi)) \\
&{} = \frac{sin(\omega \pi)}{\omega} - j \cdot \frac{2sin^2(\frac{\omega \pi}{2})}{\omega}
\end{split}
\end{equation}
\pagebreak

Sprowadziliśmy wzór do postaci $ a - j \cdot b $, korzystając przy tym z wyżej podanych własności. Zapiszmy teraz częstotliwość amplitudową, którą wyrażamy wzorem $ |\hat{f} (\omega)| = \sqrt{a^2 + b^2} $:
\begin{equation}
|\hat{f} (\omega)| = \sqrt{ \Bigg(\frac{sin(\omega \pi)}{\omega}\Bigg)^2 + \Bigg(\frac{2sin^2(\frac{\omega \pi}{2})}{\omega}\Bigg)^2 } = \frac{sin(\omega \pi)^2}{\omega^2} + \Bigg(\frac{4 \cdot sin(\frac{\omega \pi}{2})^2}{\omega^2}\Bigg)^\mathlarger{\frac{1}{2}}
\label{czestotliwosciowa}
\end{equation}
\begin{figure}[h]
\centering
\includegraphics[width=10cm]{widmowe1}
\end{figure}
\[
\text{Częstotliwość amplitudowa sygnału nr 1}
\] \\

Obliczmy także częstotliwość fazową za pomocą wzoru $ arg \hat{f} (\omega) = arctg(\frac{b}{a}) $:
\begin{equation}
arg \hat{f} (\omega) = arctg \Bigg(\frac{\frac{-2sin^2(\frac{\omega \pi}{2})}{\omega}}{\frac{sin(\omega \pi)}{\omega}}\Bigg) = arctg \Bigg(\frac{-2sin^2(\frac{\omega \pi}{2})} {sin(\omega \pi)}\Bigg)
\label{fazowa}
\end{equation}
\pagebreak

\begin{figure}[h]
\centering
\includegraphics[width=10cm]{fazowe1}
\end{figure}
\[
\text{Częstotliwość fazowa sygnału nr 1}
\] \\

A także energię z wzoru $ \mathlarger{\int_{-\infty}^{\infty}} f(t)^2 dt $:
\begin{equation}
E = \int_{0}^{\pi} 1^2 dt = t \ \Bigg|^\pi_0 = \pi
\label{energia}
\end{equation}

Oraz energię z wzoru Parsevala (korzystając przy tym z funkcji $ vpaintegral $ w programie Matlab): \\
\begin{equation}
E = \int_{-\infty}^{\infty} |f(t)|^2 dt = \frac{1}{\pi} \int_{0}^{\infty} |\hat{f}(\omega)|^2 d\omega = \frac{1}{\pi} \int_{0}^{1} \frac{sin(\omega \pi)}{\omega} - j \cdot \frac{2sin^2(\frac{\omega \pi}{2})}{\omega} d\omega
\label{parseval}
\end{equation}

Energia jest równa (przybliżone do 6 cyfr po przecinku) \\ $ \approx $ 
0.405669 - 1.578752i.
\pagebreak

\section{Sygnał nr 2}
Drugi sygnał wygląda następująco:
\begin{figure}[h]
\centering
\includegraphics[width=13cm]{sygnal2}
\end{figure} \\

Możemy go zapisać w poniższej postaci:
\[
    f(t) =\left\{
                \begin{array}{ll}
                  \frac{t}{\pi}, dla\ t <0,\pi> \\
                  0, dla\ t \in reszty
                \end{array}
\label{sygnal2klamra}
\tag{2.1}
              \right.
\]

\setcounter{liczba1}{2}
\setcounter{equation}{1}

Znów sprawdzamy warunki Dirichleta. Pierwsza jest bezwględna całkowalność (wzór \ref{warunek1}):
\begin{equation}
\int_{0}^{\pi} \frac{t}{\pi} dt = \frac{1}{\pi} \cdot \frac{1}{2} \cdot t^2 \ \Bigg|^\pi_0 = \frac{\pi}{2} - 0 = \frac{\pi}{2}
\end{equation} \\

Warunek ten jest więc spełniony, podobnie jak drugi i trzeci. Transformata (wzór \ref{transformata}) dla sygnału nr 2:
\begin{equation}
\begin{split}
\int_{0}^{\pi} \frac{t}{\pi} \cdot e^{-j \omega t} dt
&{} = \frac{1}{\pi} \cdot \int_{0}^{\pi} t \cdot e^{-j \omega t} dt \\
&{} = \frac{1}{\pi} \cdot \Bigg[\Big(e^{-j \omega \pi} \cdot (\frac{\pi}{-j \omega} - \frac{1}{(\-j \omega)^2})\Big) - \Big(e^{-j \omega 0} \cdot (\frac{0}{-j \omega} - \frac{1}{(\-j \omega)^2})\Big)\Bigg] \\
&{} = \frac{1}{\pi} \cdot \Bigg[\Big(e^{-j \omega \pi} \cdot (\frac{\pi}{-j \omega} - \frac{1}{(\-j \omega)^2})\Big) + \frac{1}{(\-j \omega)^2}\Bigg] \\
&{} = \frac{1}{\pi} \cdot \Bigg[\Big(e^{-j \omega \pi} \cdot (\frac{\pi}{-j \omega} + \frac{1}{\omega^2})\Big) - \frac{1}{\omega^2}\Bigg] = \ \cdots
\end{split}
\end{equation}

Należy sprowadzić transformatę do postaci $ a + j \cdot b $, by policzyć widmo częstotliwościowe dla sygnału drugiego za pomocą wzorów nr \ref{czestotliwosciowa} i \ref{fazowa}, oraz energię z wzoru Parsevala (nr \ref{parseval}). \\

\begin{equation}
\cdots
\end{equation} \\

Energia sygnału nr 2 ze wzoru \ref{energia} = 
\begin{equation}
E = \int_{0}^{\pi} \Big(\frac{t}{\pi}\Big)^2 dt = \frac{1}{\pi^2} \cdot \frac{1}{3} t^3 \ \Bigg|_{0}^{\pi} = \frac{1}{\pi^2} \cdot \frac{\pi^3}{3} = \frac{\pi}{3}
\end{equation} \\

\end{document}