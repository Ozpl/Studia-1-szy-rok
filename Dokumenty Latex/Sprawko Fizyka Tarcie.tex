\documentclass[11pt,a4paper]{article}

\usepackage{polski}
\usepackage[utf8]{inputenc}
\usepackage{amsmath}
\usepackage{mathrsfs}
\usepackage{wrapfig}
\usepackage{textcomp}
\usepackage{tabto}

\title{TARCIE}
\author{Bartłomiej Grządziel, Przemysław Niemiec, Piotr Zakrzewski \\ AiR, semestr 2, L2}
\date{19.03.2018r., prowadzący dr inż. Marek Wichtowski}

\begin{document}
\maketitle


\section{Wstęp teoretyczny}
\hspace*{8mm} Tarciem nazywamy opór ruchu, który oddziałuje na dwa stykające się ciała i który przeciwdziała ich względnemu ruchowi. Siła tarcia ma kierunek styczny do powierzchni zetknięcia tych ciał i działa na każde z nich. Jej zwrot jest przeciwny do zwrotu prędkości względem drugiego ciała. \\
\hspace*{8mm} Źródłem siły tarcia są oddziaływania elektrostatyczne między atomami lub cząsteczkami stykających się ciał. O sile tarcia statycznego mówimy, gdy ciała nie przesuwają się względem siebie. Jej wartość maksymalna równa jest najmniejszej wartości siły zewnętrznej, która po przyłożeniu do ciała spowoduje jego ruch. Jeżeli ciała poruszają się względem siebie, to między ich powierzchniami działają siły tarcia kinetycznego.


\section{Cel ćwiczenia}
\hspace*{8mm} Celem ćwiczenia jest poznanie przez studentów zasad działania tarcia w praktyce. Podczas jego wykonywania studenci mają szansę zaobserwować oraz opisać zależności pomiędzy siłą tarcia, a takimi wartościami jak kąt nachylenia równi, masa używanych w doświadczeniu przedmiotów, czy rodzaj ich powierzchni.

\section{Układ doświadczalny}
$ \cdot $ równia pochyła z możliwością regulacji kąta jej nachylenia \\
$ \cdot $ bloczek obrotowy na końcu równi, pozwalający na zawieszenie obok równi masy powodującej ruch zamocowanego klocka w górę po równi \\
$ \cdot $ urządzenie elektroniczne pozwalające na dokładniejsze zmierzenie kąta równi \\
$ \cdot $ waga elektroniczna \\
$ \cdot $ klocki o różnych masach oraz rodzajach powierzchni \\
$ \cdot $ linka
\pagebreak

\section{Przebieg doświadczenia}
1. Pomiary wykonać dla trzech klocków. \\
2. Wykonać pomiar masy klocka $m$. \\
3. Zanotować wymiary i rodzaj powierzchni dolnej klocka. \\
4. Położyć klocek i zmierzyć min. 3-krotnie kąt nachylenia równi, przy którym odbywa się ruch jednostajny (jeśli wartości się różnią oszacować błąd przypadkowy). Obliczyć wartość średnią kąta nachylenia ($\alpha_{sr}$), a następnie wg wzoru obliczyć współczynnik tarcia. \\
5. Wybrać i podwiesić ciężarki, zanotować masę użytych ciężarków $M$. \\
6. Wyznaczyć kąt nachylenia równi $\alpha$ przy którym odbywa się ruch jednostajny (min. 3 pomiary). Obliczyć współczynnik $f_k$. \\
7. Zapisać wyniki z podaniem niepewności pomiarowej i porównać wyniki. \\ \\
\hspace*{5mm} Tabela pomiarowa 1:
Przypadek nr 1. Klocek metalowy, rodzaj powierzchni dolnej taki sam jak materiał całego klocka.
\begin{center}
 \begin{tabular}{||c c c c c c||} 
 \hline
 Nr pom. & m[g] & $\alpha$[deg] & $\alpha_{sr}$[deg] & $\Delta \alpha$ & $f_k$ \\ [1.5ex] 
 \hline\hline
 1 & 195 & 17, 16, 16.5 & 16.5 & 0.5774 & 0.2962\\ 
 \hline
 2 & 169  & 17, 17.5, 18 & 17.5 & 0.5774 & 0.3153 \\
 \hline
 3 & 283 & 19.8, 20.2, 20.4 & 20.2 & 0.3528 & 0.3679 \\
 \hline
\end{tabular}
\end{center}

Tabela pomiarowa 2:
Przypadek nr 2a. Klocek drewniany, rodzaj powierzchni dolnej taki sam jak materiał całego klocka.
\begin{center}
 \begin{tabular}{||c c c c c c c||} 
 \hline
 Nr pom. & m[g] & $\alpha$[deg] & $\alpha_{sr}$[deg] & $\Delta \alpha$ & M[g] & $f_k$ \\ [1.5ex] 
 \hline\hline
 1 & 195 & 14, 14, 14 & 14 & 0 & 96 & 0.2581\\ 
 \hline
 2 & 169  & 4, 4.2, 4.4 & 4.2 & 0.2309 & 60 & 0.2826 \\
 \hline
 3 & 283 & 11, 11, 11 & 11 & 0 & 143 & 0.3204 \\
 \hline
\end{tabular}
\end{center}

Tabela pomiarowa 3:
Przypadek nr 2b. Klocek z brązu, powierzchnia dolna klocka to wykładzina.
\begin{center}
 \begin{tabular}{||c c c c c c c||} 
 \hline
 Nr pom. & m[g] & $\alpha$[deg] & $\alpha_{sr}$[deg] & $\Delta \alpha$ & M[g] & $f_k$ \\ [1.5ex] 
 \hline\hline
 1 & 195 & 43, 42.5, 43.5 & 43 & 0.5774 & 96 & 0.2594\\ 
 \hline
 2 & 169  & 39, 39, 39 & 39 & 0 & 60 & 0.3529 \\
 \hline
 3 & 283 & 45, 44, 46 & 45 & 1.1547 & 143 & 0.2854 \\
 \hline
\end{tabular}
\end{center}
\pagebreak

\section{Informacje dodatkowe}
\hspace*{8mm} Wszystkie obliczenia wykonano przy użyciu programu Matlab. \\ \\
Podczas wykonywania ćwiczenia wykorzystano poniższe wzory: \vspace{3mm} \\
Wzór na średnią - $ X \approx \bar{X} = \frac{1}{n} \sum_{i=1}^{n}X_i $ \\
Wzór na niepewność standardową - $ u_a(X) = \sqrt{\frac{\sum_{i=1}^{n} (X_i - \bar{X})^2}{n(n-1)}} $ \vspace{3mm} \\
Wzór na współczynnik tarcia w pierwszym przypadku - $ f_k = tg \alpha $ \vspace{3mm} \\ 
Wzór na współczynnik tarcia w drugim (2a) przypadku - $ f_k = \frac{M}{m cos\alpha} - tg \alpha $ \vspace{3mm} \\
Wzór na współczynnik tarcia w drugim (2b) przypadku - $ f_k = tg \alpha - \frac{M}{m cos \alpha} $


\section{Wnioski}
\hspace*{8mm}Tarcie jest zagadnieniem, którego zbadania wymaga wykonania odpowiednich pomiarów. \\
\hspace*{8mm}Siła tarcia w dużej mierze zależy od powierzchni użytych przedmiotów. Najmniejsze tarcie posiadają materiały określane przez nas jako gładkie.  W celu otrzymania jak najdokładniejszych wyników, każdy pomiar należy powtórzyć minimum trzy razy.
\end{document}